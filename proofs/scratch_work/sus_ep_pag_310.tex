\documentclass[11pt]{article}
\usepackage{amsmath, amssymb, amsthm}
\usepackage[margin=1in]{geometry}
\usepackage{xcolor}

% Scratch work commands
\newcommand{\scratch}[1]{\textcolor{blue}{#1}}
\newcommand{\idea}[1]{\textcolor{red}{\textbf{Idea:} #1}}
\newcommand{\question}[1]{\textcolor{orange}{\textbf{Q:} #1}}
\newcommand{\note}[1]{\textcolor{gray}{\textit{Note: #1}}}

% No page numbers for scratch work
\pagestyle{empty}

\begin{document}
	
	
**Theorem:** For the sequence $b_1, b_2, b_3, \ldots$ defined by $b_1 = 4$, $b_2 = 12$, and $b_k = b_{k-2} + b_{k-1}$ for every integer $k \geq 3$, we have that $b_n$ is divisible by 4 for every integer $n \geq 1$.

**Proof:** We prove this by strong mathematical induction.

Let $P(n)$ be the statement "$b_n$ is divisible by 4."

**Base cases:** We verify $P(1)$ and $P(2)$.
- $P(1)$: $b_1 = 4 = 4 \cdot 1$, so $b_1$ is divisible by 4.
- $P(2)$: $b_2 = 12 = 4 \cdot 3$, so $b_2$ is divisible by 4.

**Inductive step:** Let $k \geq 2$ be an arbitrary integer. Assume that $P(j)$ is true for all integers $j$ with $1 \leq j \leq k$. We will prove that $P(k+1)$ is true.

By the recurrence relation, we have:
$$b_{k+1} = b_{(k+1)-2} + b_{(k+1)-1} = b_{k-1} + b_k$$

Since $k \geq 2$, we have $k-1 \geq 1$ and $k \geq 1$, so both $b_{k-1}$ and $b_k$ fall within our inductive hypothesis. Therefore, by our assumption, both $b_{k-1}$ and $b_k$ are divisible by 4.

This means we can write $b_{k-1} = 4m$ and $b_k = 4n$ for some integers $m$ and $n$.

Therefore:
$$b_{k+1} = b_{k-1} + b_k = 4m + 4n = 4(m + n)$$

Since $m + n$ is an integer, we conclude that $b_{k+1}$ is divisible by 4, which proves $P(k+1)$.

**Conclusion:** By the principle of strong mathematical induction, $P(n)$ is true for all integers $n \geq 1$. Therefore, $b_n$ is divisible by 4 for every integer $n \geq 1$. $\blacksquare$
	
\end{document}